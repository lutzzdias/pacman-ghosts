\documentclass[12pt]{article}

% page setup
\topmargin 0.0cm
\oddsidemargin 0.2cm
\textwidth 16cm 
\textheight 23cm
\footskip 1.0cm

% title setup
\title{Entrega Intermediária}

\author{
Pedro Daniel Albuquerque Fernandes,$^{1}$ Thiago Lütz Dias$^{2}$\\
\\
\normalsize{$^{1}$uc20192187722@student.uc.pt - 2019218772}\\
\normalsize{$^{2}$lutzdias@student.uc.pt - 2023171576}\\
\small{PL6}\\
}

% do not show date
\date{}



%! END OF PREAMBLE

\begin{document} 

\maketitle 

\section*{Inky}

O Inky é o fantasma com comportamento mais simples e imprevisível: sempre que chega a uma interceção ele escolhe uma direção aleatória. Caso a direção escolhida seja oposta à que estava anteriormente, uma nova direção é escolhida, já que voltar para trás não é permitido.

\paragraph*{Perceções}
Inky precisa saber algumas coisas sobre seu estado atual:
\begin{itemize}
    \item O modo (chase ou frightened)
    \item Quais direções não são paredes
    \item A direção que estava antes de chegar à interceção
\end{itemize}

Ele não precisa saber a posição do pacman, a sua distância para o pacman, nem qualquer outra informação relacionada a qualquer outro elemento do jogo.

\paragraph*{Ação} 
Para o Inky, a sua ação é escolher uma direção válida. Para definir se a direção é válida ou não, ele faz uso de suas perceções para verificar se há uma parede ou se a direção é a oposta à anterior.
\begin{itemize}
    \item A direção selecionada não é uma parede $\land$ Inky não está indo para a direção oposta = Inky altera sua direção para a selecionada.

    \item A direção selecionada é uma parede $\lor$ Inky está indo para a direção oposta = Inky seleciona outra direção aleatoriamente e realiza as verificações novamente.
\end{itemize}

\section*{Blinky}

O Blinky tem um comportamento de perseguição para a posição do Pacman. Este comportamento é possível obter através da função \textit{getPacmanPosition} que permite ver saber onde está o Pacman, após sabermos a posição alvo podemos calcular qual é a direção que permite ao Blinky chegar mais rápido ao Pacman.

\paragraph*{Perceções}

Blinky precisa saber algumas coisas sobre seu estado atual:

\begin{itemize}
    \item O modo (chase ou frightened)
    \item Quais direções não são paredes
    \item A direção que estava antes de chegar à interceção
    \item A posição atual do Pacman
\end{itemize}

A distância entre si e o Pacman é calculada com base na posição de ambos. A direção do pacman é irrelevante para o Blinky.

\paragraph*{Ação} 
Com base na distância calculada, descobre-se a distância resultante de selecionar cada uma das direções possíveis. A partir dessas informações, Blinky seleciona a direção que minimiza a distância entre si e o Pacman.

\begin{itemize}
    \item A direção selecionada não é uma parede $\land$ Blinky não está indo para a direção oposta = Blinky altera sua direção para a selecionada.
    
    \item A direção selecionada é uma parede $\lor$ Blinky está indo para a direção oposta = Blinky seleciona a segunda melhor direção e realiza as verificações novamente.
\end{itemize}

\section*{Pinky}
O Pinky tem um comportamento semelhante ao Blinky, porém este possui um comportamento de emboscada, ou seja, quando no Blinky somente precisávamos de saber a posição do Pacman, para o Pinky também precisamos de saber a direção que o Pacman se dirige, esta direção é possível de se obter através da função \textit{getPacmanDirection}. Com base nessas informações, podemos calcular a posição 4 células à frente do Pacman. Caso a direção do Pacman seja este, adiciona-se 4 valores, caso seja oeste, subtrai-se 4 valores. O mesmo é feito caso o Pacman esteja se movendo na vertical (norte e sul). Esse cálculo é feito da seguinte maneira: 
\[ pacmanPosition + 4 * pacmanDirection\]

\paragraph*{Perceções}
Pinky precisa saber algumas coisas sobre seu estado atual:

\begin{itemize}
    \item O modo (chase ou frightened)
    \item Quais direções não são paredes
    \item A direção que estava antes de chegar à interceção
    \item A posição atual do Pacman
    \item A direção atual do Pacman
\end{itemize}

\paragraph*{Ação} 
Com base na posição alvo calculada, descobre-se a distância resultante de selecionar cada uma das direções possíveis. A partir dessas informações, Pinky seleciona a direção que minimiza a distância entre si e o destino alvo.

Portanto, a decisão final seria algo parecido com:
\begin{itemize}
    \item A direção selecionada não é uma parede $\land$ Pinky não está indo para a direção oposta = Pinky altera sua direção para a selecionada.
    
    \item A direção selecionada é uma parede $\lor$ Pinky está indo para a direção oposta = Pinky seleciona a segunda melhor direção e realiza as verificações novamente.
\end{itemize}

\section*{Clyde}
O Clyde é um fantasma com fobia dos outros fantasmas, ou seja, tenta sempre maximizar a sua distância para com o fantasma mais próximo. Com base nas funções \textit{getClosestGhostPosition} conseguimos definir a posição do fantasma mais próximo. Através dessa informação, podemos selecionar uma direção para Clyde que maximiza a distância entre si e o fantasma mais próximo.

\paragraph*{Perceções}
Clyde precisa saber algumas coisas sobre seu estado atual:

\begin{itemize}
    \item O modo (chase ou frightened)
    \item Quais direções não são paredes
    \item A direção que estava antes de chegar à interceção
    \item A posição do fantasma mais próximo
\end{itemize}

Como seu comportamento é definido completamente pelos outros fantasmas, ele não precisa saber nada sobre o Pacman.

\paragraph*{Ação} 
Suas ações, portanto, são resultado da análise feita em relação ao fantasma mais próximo:

\begin{itemize}
    \item A direção selecionada não é uma parede $\land$ Clyde não está indo para a direção oposta = Clyde altera sua direção para a selecionada.
    
    \item A direção selecionada é uma parede $\lor$ Clyde está indo para a direção oposta = Clyde seleciona a segunda melhor direção e realiza as verificações novamente.
\end{itemize}

\end{document}
