\documentclass[12pt]{article}

% page setup
\topmargin 0.0cm
\oddsidemargin 0.2cm
\textwidth 16cm 
\textheight 21cm
\footskip 1.0cm

% title setup
\title{Entrega Intermediária} 

\author{
    Pedro Daniel Fernandes,$^{1}$ Thiago Lütz Dias$^{2}$\\
\\
\normalsize{$^{1}$uc20192187722@student.uc.pt - 2019218772}\\
\normalsize{$^{2}$lutzdias@student.uc.pt - 2023171576}\\
}

% do not show date
\date{}



%%%%%%%%%%%%%%%%% END OF PREAMBLE %%%%%%%%%%%%%%%%



\begin{document} 

\maketitle 

\section*{Inky}

O Inky é o fantasma com comportamento mais simples e imprevisível: sempre que
chega a uma interceção ele escolhe uma direção aleatória. Caso a direção
escolhida seja oposta à que estava anteriormente, uma nova direção é escolhida,
já que voltar para trás não é permitido.

Portanto, o \textit{Sistema de Produção} do Inky não requer quase nenhuma
perceção. Já as ações são as mesmas para todos os fantasmas.

\paragraph*{Perceções}
Inky precisa saber algumas coisas sobre seu estado atual:
\begin{itemize}
    \item O modo (chase ou frightened)
    \item Quais direções são válidas (e.g não há uma parede)
    \item A direção que estava antes de chegar à interceção
\end{itemize}
Ele não precisa saber a posição do pacman, distância de si para o pacman nem
qualquer outra informação relacionada a qualquer outro elemento do jogo.

\paragraph*{Ação}
A ação é quase a mesma para todos, com alguns ajustes relacionados ao seu
comportamento em relação às perceções disponíveis. Para o Inky, sua ação é
escolher uma direção válida. Para definir se uma direção é válida ou não, ele
faz uso de suas perceções para verificar se há uma parede ou se a direção é a
oposta à anterior.

\section*{Blinky}

\section*{Pinky}

\section*{Clyde}

\end{document}
